\Lecture{Jayalal Sarma}{Oct 21, 2020}{19}{Multidimentional Ramsey numbers}{Reetwik Das}{$\alpha$}{JS}

\section{Multidimentional Ramsey numbers}
\begin{definition}
$R_k(S_1,S_2,..S_k)$ is mininum number $n$ such that any $k$-edge coloring of $K_n$ must have $K_{S_i}$ of color $i$ for some $i \in \{1,2..k\}$
\end{definition}
We need to show that why should exist $R_k(S_1,S_2,..S_k)$.
\begin{claim}
$$R_k(S_1,S_2,..S_k) \leq R_{k/2}(R(S_1,S_2), R(S_3,S_4).... R(S_{k-1},S_k))$$
\end{claim}
\begin{proof}
Let $n =  R_{k/2}(R(S_1,S_2), R(S_3,S_4).... R(S_{k-1},S_k))$\\
Consider $K_n$ and any K-edge coloring of the edges of $K_n$\\

We need to show that $\exists S_1$ clique of color 1 or $S_2$ clique of color 2.\\
Consider colors paired up and rename them $\{1,2\} = 1, \{3,4\} = 2, ... \{k-1,k\} = k/2$ \\
By the definition of $R_{k/2}$ we are guaranteed $\exists i$ such that $\exists$ a clique of size $R(S_{2i-1},S_{2i})$ of color $i$.\\

After we uninterpret the color $i$ as the original pair of colors we get a 2-coloring of the clique that we have $R(S_{2i-1},S_{2i})$\\
By the definition of $R_2$ we know that $\exists$ a $S_{2i-1}$ clique of color $2i-1$ or $S_{2i}$ clique of color $2i$.
\end{proof}

\section{Fermat's last theorem}
We know from Pythagoras theorem that $x^2 +y^2 = z^2$ has integral solutions. But we want to know if this equation has any integral solutions for any power greater than $2$.
\begin{theorem}
$x^n +y^n = z^n$ doesn't have any integral solutions $\forall n>2$.
\end{theorem}

\Lecture{Jayalal Sarma}{Oct 22, 2020}{20}{Finite fields}{Reetwik Das}{$\alpha$}{JS}

\subsection{Finite fields}
$x^n +y^n = z^n$ does have integeral solutions for finite fields such as for $Z_p$.\\
$Z_p = \{0,1,....p-1\}$ and addition and multiplication are $modulo$ $p$ within this field.\\

\begin{definition}
$Z_p^*$ is a cyclic group $\{1,2,3....p-1\}$
\end{definition}

\begin{claim}
If $p$ is a prime then $Z_p^*$ is generated by a single element, and the element is known as the generator.
\end{claim}

Fermat's last theorem is completely algebraic to connect it to coloring we need a tool.
\begin{theorem}
\textbf{Schur's theorem :} If $r \geq 0$ positive integer then. $\exists$ integer $S(r)$ such that if we color $\{1,2,....S(r)\}$ vertices with $r$ colors then $\exists x,y,z$ in the set and $x+y =z$
\end{theorem}
\begin{proof}
Given an $r$, Let $S(r) = R_r(3,3,3,...3)$\\
Consider $K_n$, $n = S(r)$ by the definition if we color the edges of $K_n$ using $r$ colors then we are guaranteed a monochromatic $K_3$.\\
We are given a coloring $\{1,2,...S(r)\}$\\
Define a coloring for edges in $K_n$.\\
Associate vertices of $K_n$ with elements in $\{1,2,...S(r)\}$ \\
$\forall a,b\in V$ the color of edge $(a,b) $ = color of $|a-b|$\\

Let $\{\alpha, \beta,\gamma\}$ be the vertices of the monochromatic triangle.
Let $x = \alpha - \beta$, $y = \beta - \gamma$ and $z = \alpha - \gamma$ then $x,y,z$ have the same color.
It also satifies the equation $x+y =z$.
 
\end{proof}

\begin{theorem}
$\forall m \exists q$ such that $\forall p\geq q$ in $Z_p$ ($p$ is a prime) \\
$x^m +y^m = z^m$ has a solution.
\end{theorem}
\begin{proof}
Given $m$ from $ x^m +y^m = z^m$\\
$p=q=S(m)+1$ by Schur's theorem any coloring of $\{1,2...q\}$ must have a triplet $a+b = c$.\\
$Z_p = {0,1,2...q-1}$\\
Let $g$ be the generator of $Z_p$ then every non-zero element in $Z_p = g^k$ for some $k$.\\

Assign the coloring $\{1,2...q\}$ as follows :\\
$\forall x \in Z_p^*,  x=g^{mi+j}$ and $color(x)= j = k (mod m)$\\

By Schur's theorem, $\exists a,b,c$ such that $a+b=c$ and all have the same color.\\
$$g^{mi_a+j} + g^{mi_b+j} = g^{mi_c+j}$$
$$(g^{i_a})^m + (g^{i_b})^m = (g^{i_c})^m$$
and we have the solution for $x^m +y^m = z^m$. 
\end{proof}

\subsection{Lower bounds for Ramsey numbers}
\begin{claim}
$$\forall k, R(k,k) > 2^{k/2}$$
\end{claim}
\begin{proof}
Suffices to show that $n = 2^{k/2}$, $\exists$ a 2-coloring of the edges of $K_n$ such that there is no monochromatic $K_k$ in it.\\

Fix $m = 2^{k/2}$ there are $n\choose 2$ many edges.\\
A coloring is said to be bad if $\exists$ no monochromatic $K_k$ in it.\\

\textbf{Probabilistic method :}\\
For every edge, assign red/blue color with probability 1/2 each.\\
if we show that the probability[coloring is bad]$ > 0$ then this means $\exists$ a bad coloring.\\

Suffices to show that the Pr[coloring is good] $<1$\\
Pr[$\exists K_k$ which is monochromatic] $\leq \Sigma_{S \subseteq K_k,|S|=k}$ Pr[S is monochromatic]\\
$ = {n\choose k}$ Pr[S is monochromatic]
$$ = {n\choose k} \frac{2}{2^{k\choose 2}}$$
$$= {n\choose k} 2^{1-{k\choose2}}$$
$$ =\frac{n(n-1)...(n-k+1)}{k!} \frac{2^{1+k/2}}{2^{k^2/2}}$$
$$ \leq \frac{n^k}{k!} \frac{2^{1+k/2}}{2^{k^2/2}}$$
$$= \frac{2^{1+k/2}}{k!} < 1$$
\end{proof}
